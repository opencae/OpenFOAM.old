%#! platex UserGuideJa
\begin{longtable}{lX}
 \multicolumn{2}{l}{状態方程式 ---
\index{equationOfState@\string\OFclass{equationOfState}!ライブラリ}%
\index{ライブラリ!equationOfState@\string\OFclass{equationOfState}}%
 \OFclass{equationOfState}} \\
 \hline
 \tblstrut
\index{adiabaticPerfectFluid@\OFclass{adiabaticPerfectFluid}!モデル}%
\index{モデル!adiabaticPerfectFluid@\OFclass{adiabaticPerfectFluid}}%
 \OFclass{adiabaticPerfectFluid} &
 断熱完全気体の状態方程式 \\
\index{icoPolynomial@\OFclass{icoPolynomial}!モデル}%
\index{モデル!icoPolynomial@\OFclass{icoPolynomial}}%
 \OFclass{icoPolynomial} &
 液体などの非圧縮性流体に対する多項式の状態方程式 \\
\index{perfectFluid@\OFclass{perfectFluid}!モデル}%
\index{モデル!perfectFluid@\OFclass{perfectFluid}}%
 \OFclass{perfectFluid} &
 完全気体の状態方程式 \\
\index{incompressiblePerfectGas@\OFclass{incompressiblePerfectGas}!モデル}%
\index{モデル!incompressiblePerfectGas@\OFclass{incompressiblePerfectGas}}%
 \OFclass{incompressiblePerfectGas} &
 一定の参照圧力を用いた非圧縮性気体の状態方程式.
 密度は温度と組成によってのみ変化する. \\
\index{rhoConst@\OFclass{rhoConst}!モデル}%
\index{モデル!rhoConst@\OFclass{rhoConst}}%
 \OFclass{rhoConst} &
 密度を一定とした状態方程式 \\
 \\
 \multicolumn{2}{l}{標準熱特性 ---
\index{thermo@\string\OFclass{thermo}!ライブラリ}%
\index{ライブラリ!thermo@\string\OFclass{thermo}}%
 \OFclass{thermo}} \\
 \hline
 \tblstrut
\index{eConstThermo@\OFclass{eConstThermo}!モデル}%
\index{モデル!eConstThermo@\OFclass{eConstThermo}}%
 \OFclass{eConstThermo} &
 内部エネルギ$e$とエントロピ$s$の評価を備えた,比熱$c_{\mathrm{p}}$一定のモデル \\
\index{hConstThermo@\OFclass{hConstThermo}!モデル}%
\index{モデル!hConstThermo@\OFclass{hConstThermo}}%
 \OFclass{hConstThermo} &
 エンタルピ$h$とエントロピ$s$の評価を備えた,比熱$c_{\mathrm{p}}$一定のモデル \\
\index{hPolynomialThermo@\OFclass{hPolynomialThermo}!モデル}%
\index{モデル!hPolynomialThermo@\OFclass{hPolynomialThermo}}%
 \OFclass{hPolynomialThermo} &
 $h$と$s$を評価する多項式の係数の関数により$c_{\mathrm{p}}$が評価される \\
\index{janafThermo@\OFclass{janafThermo}!モデル}%
\index{モデル!janafThermo@\OFclass{janafThermo}}%
 \OFclass{janafThermo} &
 JANAFの熱物性表の係数から$c_{\mathrm{p}}$が評価され,
 それにより$h$,$s$が評価される. \\
 \\
 \multicolumn{2}{l}{派生熱特性 ---
\index{specieThermo@\string\OFclass{specieThermo}!ライブラリ}%
\index{ライブラリ!specieThermo@\string\OFclass{specieThermo}}%
 \OFclass{specieThermo}} \\
 \hline
 \tblstrut
\index{specieThermo@\OFclass{specieThermo}!モデル}%
\index{モデル!specieThermo@\OFclass{specieThermo}}%
 \OFclass{specieThermo} &
 $c_{\mathrm{p}}$,$h$,そして/または,$s$から得られた特殊な熱特性 \\
 \\
 \multicolumn{2}{l}{輸送特性 ---
\index{transport@\string\OFclass{transport}!ライブラリ}%
\index{ライブラリ!transport@\string\OFclass{transport}}%
 \OFclass{transport}} \\
 \hline
 \tblstrut
\index{constTransport@\OFclass{constTransport}!モデル}%
\index{モデル!constTransport@\OFclass{constTransport}}%
 \OFclass{constTransport} &
 一定の輸送特性 \\
\index{polynomialTransport@\OFclass{polynomialTransport}!モデル}%
\index{モデル!polynomialTransport@\OFclass{polynomialTransport}}%
 \OFclass{polynomialTransport} &
 多項式に基づく温度依存輸送特性 \\
\index{sutherlandTransport@\OFclass{sutherlandTransport}!モデル}%
\index{モデル!sutherlandTransport@\OFclass{sutherlandTransport}}%
 \OFclass{sutherlandTransport} &
 温度依存する輸送輸送のためのSutherlandの公式 \\
 \\
 \multicolumn{2}{l}{混合特性 ---
\index{mixture@\string\OFclass{mixture}!ライブラリ}%
\index{ライブラリ!mixture@\string\OFclass{mixture}}%
 \OFclass{mixture}} \\
 \hline
 \tblstrut
\index{pureMixture@\OFclass{pureMixture}!モデル}%
\index{モデル!pureMixture@\OFclass{pureMixture}}%
 \OFclass{pureMixture} &
 不活性混合気体の一般熱物理モデル計算 \\
\index{homogeneousMixture@\OFclass{homogeneousMixture}!モデル}%
\index{モデル!homogeneousMixture@\OFclass{homogeneousMixture}}%
 \OFclass{homogeneousMixture} &
 正規化燃料質量分率$b$に基づく混合気燃焼 \\
\index{inhomogeneousMixture@\OFclass{inhomogeneousMixture}!モデル}%
\index{モデル!inhomogeneousMixture@\OFclass{inhomogeneousMixture}}%
 \OFclass{inhomogeneousMixture} &
 $b$と総燃料質量分率$f_{\mathrm{t}}$に基づく混合気燃焼 \\
\index{veryInhomogeneousMixture@\OFclass{veryInhomogeneousMixture}!モデル}%
\index{モデル!veryInhomogeneousMixture@\OFclass{veryInhomogeneousMixture}}%
 \OFclass{veryInhomogeneousMixture} &
 $b$と$f_{\mathrm{t}}$と未燃燃料質量分率$f_{\mathrm{u}}$に基づく混合気燃焼 \\
\index{basicMultiComponentMixture@\OFclass{basicMultiComponentMixture}!モデル}%
\index{モデル!basicMultiComponentMixture@\OFclass{basicMultiComponentMixture}}%
 \OFclass{basicMultiComponentMixture} &
 複数の成分に基づく基本的な混合気 \\
\index{multiComponentMixture@\OFclass{multiComponentMixture}!モデル}%
\index{モデル!multiComponentMixture@\OFclass{multiComponentMixture}}%
 \OFclass{multiComponentMixture} &
 複数の成分に基づく派生的な混合気 \\
\index{reactingMixture@\OFclass{reactingMixture}!モデル}%
\index{モデル!reactingMixture@\OFclass{reactingMixture}}%
 \OFclass{reactingMixture} &
 熱力学と反応スキームを用いた混合気燃焼 \\
\index{egrMixture@\OFclass{egrMixture}!モデル}%
\index{モデル!egrMixture@\OFclass{egrMixture}}%
 \OFclass{egrMixture} &
 排気ガス再循環の混合気 \\
\index{singleStepReactingMixture@\OFclass{singleStepReactingMixture}!モデル}%
\index{モデル!singleStepReactingMixture@\OFclass{singleStepReactingMixture}}%
 \OFclass{singleStepReactingMixture} &
 素反応を伴う混合気 \\
 \\
 \multicolumn{2}{l}{熱モデル ---
\index{thermoModel@\string\OFclass{thermoModel}!ライブラリ}%
\index{ライブラリ!thermoModel@\string\OFclass{thermoModel}}%
 \OFclass{thermoModel}} \\
 \hline
 \tblstrut
\index{hePsiThermo@\OFclass{hePsiThermo}!モデル}%
\index{モデル!hePsiThermo@\OFclass{hePsiThermo}}%
 \OFclass{hePsiThermo} &
 圧縮率$\psi$に基づく一般熱物理モデル計算 \\
\index{heRhoThermo@\OFclass{heRhoThermo}!モデル}%
\index{モデル!heRhoThermo@\OFclass{heRhoThermo}}%
 \OFclass{heRhoThermo} &
 密度$\rho$に基づく一般熱物理モデル計算 \\
\index{psiReactionThermo@\OFclass{psiReactionThermo}!モデル}%
\index{モデル!psiReactionThermo@\OFclass{psiReactionThermo}}%
 \OFclass{psiReactionThermo} &
 $\psi$に基づいて燃焼混合気のエンタルピを計算する \\
\index{psiuReactionThermo@\OFclass{psiuReactionThermo}!モデル}%
\index{モデル!psiuReactionThermo@\OFclass{psiuReactionThermo}}%
 \OFclass{psiuReactionThermo} &
 $\psi_{\mathrm{u}}$に基づいて燃焼混合気のエンタルピを計算する \\
\index{rhoReactionThermo@\OFclass{rhoReactionThermo}!モデル}%
\index{モデル!rhoReactionThermo@\OFclass{rhoReactionThermo}}%
 \OFclass{rhoReactionThermo} &
 $\rho$に基づいて燃焼混合気のエンタルピを計算する \\
\index{heheupsiReactionThermo@\OFclass{heheupsiReactionThermo}!モデル}%
\index{モデル!heheupsiReactionThermo@\OFclass{heheupsiReactionThermo}}%
 \OFclass{heheupsiReactionThermo} &
 未燃ガスおよび燃焼混合気のエンタルピを計算する \\
\end{longtable}
