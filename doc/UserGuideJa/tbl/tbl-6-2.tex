%#! uplatex UserGuideJa
\begin{tabularx}{\textwidth}{lX}
 環境変数 & 説明とオプション \\
 \hline
 \tblstrut
\index{CEI HOME@\string\OFenv{CEI\_HOME}!かんきょうへんすう@環境変数}%
\index{かんきょうへんすう@環境変数!CEI HOME@\string\OFenv{CEI\_HOME}}%
 \OFenv{\$CEI\_HOME} &
     EnSightがインストールされるパス(例:\OFpath{/usr/local/ensight})はデフォルトでシステムパスに加わる \\
\index{CEI ARCH@\string\OFenv{CEI\_ARCH}!かんきょうへんすう@環境変数}%
\index{かんきょうへんすう@環境変数!CEI ARCH@\string\OFenv{CEI\_ARCH}}%
 \OFenv{\$CEI\_ARCH} &
     \OFpath{\$CEI\_HOME/ensight74/machines}の
     マシンディレクトリ名に対応する名前から選択したマシン構造.
     デフォルト設定では\verb|linux_2.4|と\verb|sgi_6.5_n32|を含む \\
\index{ENSIGHT7 READER@\string\OFenv{ENSIGHT7\_READER}!かんきょうへんすう@環境変数}%
\index{かんきょうへんすう@環境変数!ENSIGHT7 READER@\string\OFenv{ENSIGHT7\_READER}}%
 \OFenv{\$ENSIGHT7\_READER} &
     EnSightがユーザの定義したlibuserd-foam読込みライブラリを探すパス,
     デフォルトでは\OFenv{\$FOAM\_LIBBIN}に設定 \\
\index{ENSIGHT7 INPUT@\string\OFenv{ENSIGHT7\_INPUT}!かんきょうへんすう@環境変数}%
\index{かんきょうへんすう@環境変数!ENSIGHT7 INPUT@\string\OFenv{ENSIGHT7\_INPUT}}%
 \OFenv{\$ENSIGHT7\_INPUT} &
     デフォルトでは\texttt{dummy}に設定 \\
 \hline
\end{tabularx}
