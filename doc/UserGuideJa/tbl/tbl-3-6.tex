%#! uplatex UserGuideJa
\begin{longtable}{lX}
 \multicolumn{2}{l}{前処理} \\
 \hline
 \tblstrut
\index{applyBoundaryLayer@\OFtool{applyBoundaryLayer}!ユーティリティ}%
\index{ユーティリティ!applyBoundaryLayer@\OFtool{applyBoundaryLayer}}%
 \OFtool{applyBoundaryLayer} &
 $1/7$乗則に基づいて,速度場と乱流場に簡易的な境界層モデルを適用する. \\
\index{applyWallFunctionBoundaryConditions@\OFtool{applyWallFunctionBoundaryConditions}!ユーティリティ}%
\index{ユーティリティ!applyWallFunctionBoundaryConditions@\OFtool{applyWallFunctionBoundaryConditions}}%
 \OFtool{applyWallFunctionBoundaryConditions} &
 OpenFOAMのRASケースを,新しい(バージョン1.6の)壁関数を使うように更新する. \\
\index{boxTurb@\OFtool{boxTurb}!ユーティリティ}%
\index{ユーティリティ!boxTurb@\OFtool{boxTurb}}%
 \OFtool{boxTurb} & 与えられたエネルギスペクトルに適合し,
 連続の式を満す乱流のboxを生成する \\
\index{changeDictionary@\OFtool{changeDictionary}!ユーティリティ}%
\index{ユーティリティ!changeDictionary@\OFtool{changeDictionary}}%
 \OFtool{changeDictionary} &
 ディクショナリのエントリを変更するユーティリティ.
 たとえば,フィールドと\OFpath{polyMesh/boundary}ファイルの
 パッチタイプを変更するときなどに使える. \\
\index{createExternalCoupledPatchGeometry@\OFtool{createExternalCoupledPatchGeometry}!ユーティリティ}%
\index{ユーティリティ!createExternalCoupledPatchGeometry@\OFtool{createExternalCoupledPatchGeometry}}%
 \OFtool{createExternalCoupledPatchGeometry} &
 \OFboundary{externalCoupled}境界条件で使うためのパッチ形状(点および面)を生成する. \\
\index{dsmcInitialise@\OFtool{dsmcInitialise}!ユーティリティ}%
\index{ユーティリティ!dsmcInitialise@\OFtool{dsmcInitialise}}%
 \OFtool{dsmcInitialise} &
 初期化ディクショナリ\OFdictionary{system/dsmcInitialise}に従って,
 \OFtool{dsmcFoam}用にケースを初期化する \\
\index{engineSwirl@\OFtool{engineSwirl}!ユーティリティ}%
\index{ユーティリティ!engineSwirl@\OFtool{engineSwirl}}%
 \OFtool{engineSwirl} & エンジン計算のために旋回流を発生させる \\
\index{faceAgglomerate@\OFtool{faceAgglomerate}!ユーティリティ}%
\index{ユーティリティ!faceAgglomerate@\OFtool{faceAgglomerate}}%
 \OFtool{faceAgglomerate} &
 形態係数放射モデルの解析用に境界面をグルーピングして,
 細かい格子から粗い格子へのマップを書き出す. \\
\index{foamUpgradeCyclics@\OFtool{foamUpgradeCyclics}!ユーティリティ}%
\index{ユーティリティ!foamUpgradeCyclics@\OFtool{foamUpgradeCyclics}}%
 \OFtool{foamUpgradeCyclics} &
 分離された周期境界のメッシュや場を更新するツール \\
\index{foamUpgradeFvSolution@\OFtool{foamUpgradeFvSolution}!ユーティリティ}%
\index{ユーティリティ!foamUpgradeFvSolution@\OFtool{foamUpgradeFvSolution}}%
 \OFtool{foamUpgradeFvSolution} &
 \OFsubdictionary{system/fvSolution::solvers}の書式を更新する簡易ツール \\
\index{mapFields@\OFtool{mapFields}!ユーティリティ}%
\index{ユーティリティ!mapFields@\OFtool{mapFields}}%
 \OFtool{mapFields} & 両ケースの時刻ディレクトリの全ての場を読み込み,
 補間し,体積場を一つのメッシュから他のメッシュにマップする.
 並列・非並列のどちらのケースでも再構築せずに実行可能 \\
\index{mapFieldsPar@\OFtool{mapFieldsPar}!ユーティリティ}%
\index{ユーティリティ!mapFieldsPar@\OFtool{mapFieldsPar}}%
 \OFtool{mapFieldsPar} & 両ケースの時刻ディレクトリの全ての場を読み込み,
 補間し,体積場を一つのメッシュから他のメッシュにマップする.
 \OFrevision*{こちらは並列版?} \\
\index{mdInitialise@\OFtool{mdInitialise}!ユーティリティ}%
\index{ユーティリティ!mdInitialise@\OFtool{mdInitialise}}%
 \OFtool{mdInitialise} &
 分子動力学 (MD) シミュレーションのフィールドを初期化する. \\
\index{setFields@\OFtool{setFields}!ユーティリティ}%
\index{ユーティリティ!setFields@\OFtool{setFields}}%
 \OFtool{setFields} & ディクショナリによって,選択されたセル・パッチのセット上に値を設定する. \\
\index{viewFactorGen@\OFtool{viewFactorGen}!ユーティリティ}%
\index{ユーティリティ!viewFactorGen@\OFtool{viewFactorGen}}%
 \OFtool{viewFactorGen} &
 グルーピングされた面 (\OFtool{faceAgglomerate}) に基づいて,
 形態係数放射モデルで使うための形態係数を計算する. \\
\index{wallFunctionTable@\OFtool{wallFunctionTable}!ユーティリティ}%
\index{ユーティリティ!wallFunctionTable@\OFtool{wallFunctionTable}}%
 \OFtool{wallFunctionTable} & 乱流の壁関数で使用される表を生成する. \\
 \\
 \multicolumn{2}{l}{メッシュ生成} \\
 \hline
 \tblstrut
\index{blockMesh@\OFtool{blockMesh}!ユーティリティ}%
\index{ユーティリティ!blockMesh@\OFtool{blockMesh}}%
 \OFtool{blockMesh} & マルチブロック・メッシュのジェネレータ \\
\index{extrudeMesh@\OFtool{extrudeMesh}!ユーティリティ}%
\index{ユーティリティ!extrudeMesh@\OFtool{extrudeMesh}}%
 \OFtool{extrudeMesh} &
 既存のパッチやファイルから読み込んだパッチを
 (デフォルトでは面の外側へ,オプションで反転して)押し出す. \\
\index{extrude2DMesh@\OFtool{extrude2DMesh}!ユーティリティ}%
\index{ユーティリティ!extrude2DMesh@\OFtool{extrude2DMesh}}%
 \OFtool{extrude2DMesh} &
 2Dメッシュ(すべての面が2点で,前後の面がない)を読み込み,
 与えられた厚さに押し出すことで3Dメッシュをつくる. \\
\index{extrudeToRegionMesh@\OFtool{extrudeToRegionMesh}!ユーティリティ}%
\index{ユーティリティ!extrudeToRegionMesh@\OFtool{extrudeToRegionMesh}}%
 \OFtool{extrudeToRegionMesh} &
 \OFkeyword{faceZones}を個別のメッシュに(別の領域として)押し出す.
 例えば,液体の膜領域を作るために \\
\index{foamyHexMesh@\OFtool{foamyHexMesh}!ユーティリティ}%
\index{ユーティリティ!foamyHexMesh@\OFtool{foamyHexMesh}}%
 \OFtool{foamyHexMesh} &
 等角ボロノイ自動メッシュ生成ツール \\
\index{foamyHexMeshBackgroundMesh@\OFtool{foamyHexMeshBackgroundMesh}!ユーティリティ}%
\index{ユーティリティ!foamyHexMeshBackgroundMesh@\OFtool{foamyHexMeshBackgroundMesh}}%
 \OFtool{foamyHexMeshBackgroundMesh} &
 \OFtool{foamyHexMesh}で構成されたように背景メッシュを書き出し,
 \OFclass{distanceSurface}を構成する. \\
\index{foamyHexMeshSurfaceSimplify@\OFtool{foamyHexMeshSurfaceSimplify}!ユーティリティ}%
\index{ユーティリティ!foamyHexMeshSurfaceSimplify@\OFtool{foamyHexMeshSurfaceSimplify}}%
 \OFtool{foamyHexMeshSurfaceSimplify} &
 再抽出により面を単純化する. \\
\index{foamyQuadMesh@\OFtool{foamyQuadMesh}!ユーティリティ}%
\index{ユーティリティ!foamyQuadMesh@\OFtool{foamyQuadMesh}}%
 \OFtool{foamyQuadMesh} &
 等角ボロノイ2次元押し出しによる自動メッシャ \\
\index{snappyHexMesh@\OFtool{snappyHexMesh}!ユーティリティ}%
\index{ユーティリティ!snappyHexMesh@\OFtool{snappyHexMesh}}%
 \OFtool{snappyHexMesh} &
 自動分割六面体メッシャ.細分化して面にスナップする. \\
 \\
 \multicolumn{2}{l}{メッシュの変換} \\
 \hline
 \tblstrut
\index{ansysToFoam@\OFtool{ansysToFoam}!ユーティリティ}%
\index{ユーティリティ!ansysToFoam@\OFtool{ansysToFoam}}%
 \OFtool{ansysToFoam} & I-DEASから出力した
 ANSYSインプットメッシュファイルをOpenFOAM形式へ変換する \\
\index{cfx4ToFoam@\OFtool{cfx4ToFoam}!ユーティリティ}%
\index{ユーティリティ!cfx4ToFoam@\OFtool{cfx4ToFoam}}%
 \OFtool{cfx4ToFoam} & CFX~4メッシュをOpenFOAM形式へ変換する \\
\index{datToFoam@\OFtool{datToFoam}!ユーティリティ}%
\index{ユーティリティ!datToFoam@\OFtool{datToFoam}}%
 \OFtool{datToFoam} &
 datToFoamメッシュファイル内を読み,
 \OFpath{points}ファイルを出力する.
 \OFtool{blockMesh}との結合に使われる. \\
\index{fluent3DMeshToFoam@\OFtool{fluent3DMeshToFoam}!ユーティリティ}%
\index{ユーティリティ!fluent3DMeshToFoam@\OFtool{fluent3DMeshToFoam}}%
 \OFtool{fluent3DMeshToFoam} & FluentのメッシュをOpenFOAM形式に変換する \\
\index{fluentMeshToFoam@\OFtool{fluentMeshToFoam}!ユーティリティ}%
\index{ユーティリティ!fluentMeshToFoam@\OFtool{fluentMeshToFoam}}%
 \OFtool{fluentMeshToFoam} & FluentのメッシュをOpenFOAM形式に変換する.
 複数の領域と,領域の境界の処理も扱える \\
\index{foamMeshToFluent@\OFtool{foamMeshToFluent}!ユーティリティ}%
\index{ユーティリティ!foamMeshToFluent@\OFtool{foamMeshToFluent}}%
 \OFtool{foamMeshToFluent} & OpenFOAMメッシュをFluentメッシュ形式で出力する \\
\index{foamToStarMesh@\OFtool{foamToStarMesh}!ユーティリティ}%
\index{ユーティリティ!foamToStarMesh@\OFtool{foamToStarMesh}}%
 \OFtool{foamToStarMesh} &
 OpenFOAMメッシュを読み込み,
 PROSTAR (v4) のbnd/cel/vrtフォーマットに書き出す \\
\index{foamToSurface@\OFtool{foamToSurface}!ユーティリティ}%
\index{ユーティリティ!foamToSurface@\OFtool{foamToSurface}}%
 \OFtool{foamToSurface} &
 OpenFOAMのメッシュを読み込み,面のフォーマットで境界を書き出す. \\
\index{gambitToFoam@\OFtool{gambitToFoam}!ユーティリティ}%
\index{ユーティリティ!gambitToFoam@\OFtool{gambitToFoam}}%
 \OFtool{gambitToFoam} & GAMBITメッシュをOpenFOAM形式へ変換する \\
\index{gmshToFoam@\OFtool{gmshToFoam}!ユーティリティ}%
\index{ユーティリティ!gmshToFoam@\OFtool{gmshToFoam}}%
 \OFtool{gmshToFoam} & Gmshによって書かれた\OFpath{.msh}ファイルを読み込む \\
\index{ideasUnvToFoam@\OFtool{ideasUnvToFoam}!ユーティリティ}%
\index{ユーティリティ!ideasUnvToFoam@\OFtool{ideasUnvToFoam}}%
 \OFtool{ideasUnvToFoam} & I-DEAS \OFpath{unv}フォーマットのメッシュ変換 \\
\index{kivaToFoam@\OFtool{kivaToFoam}!ユーティリティ}%
\index{ユーティリティ!kivaToFoam@\OFtool{kivaToFoam}}%
 \OFtool{kivaToFoam} & KIVAグリッドをOpenFOAM形式へ変換する \\
\index{mshToFoam@\OFtool{mshToFoam}!ユーティリティ}%
\index{ユーティリティ!mshToFoam@\OFtool{mshToFoam}}%
 \OFtool{mshToFoam} & Adventureシステムによって作られた\OFpath{.msh}形式を読み込む \\
\index{netgenNeutralToFoam@\OFtool{netgenNeutralToFoam}!ユーティリティ}%
\index{ユーティリティ!netgenNeutralToFoam@\OFtool{netgenNeutralToFoam}}%
 \OFtool{netgenNeutralToFoam} &
 Netgen v4.4によって書かれたNeutralファイルフォーマットを変換する \\
\index{plot3dToFoam@\OFtool{plot3dToFoam}!ユーティリティ}%
\index{ユーティリティ!plot3dToFoam@\OFtool{plot3dToFoam}}%
 \OFtool{plot3dToFoam} &
 Plot3dメッシュ(アスキー形式)をOpenFOAM形式に変換 \\
\index{sammToFoam@\OFtool{sammToFoam}!ユーティリティ}%
\index{ユーティリティ!sammToFoam@\OFtool{sammToFoam}}%
 \OFtool{sammToFoam} & STAR-CD (v3) SAMMメッシュをOpenFOAM形式へ変換する \\
\index{star3ToFoam@\OFtool{star3ToFoam}!ユーティリティ}%
\index{ユーティリティ!star3ToFoam@\OFtool{star3ToFoam}}%
 \OFtool{star3ToFoam} & STAR-CD (v3) PROSTARメッシュをOpenFOAM形式へ変換する \\
\index{star4ToFoam@\OFtool{star4ToFoam}!ユーティリティ}%
\index{ユーティリティ!star4ToFoam@\OFtool{star4ToFoam}}%
 \OFtool{star4ToFoam} & STAR-CD (v4) PROSTARメッシュをOpenFOAM形式へ変換する \\
\index{tetgenToFoam@\OFtool{tetgenToFoam}!ユーティリティ}%
\index{ユーティリティ!tetgenToFoam@\OFtool{tetgenToFoam}}%
 \OFtool{tetgenToFoam} & tetgenにより書かれた .ele,
 .node,.faceファイルを変換する \\
\index{vtkUnstructuredToFoam@\OFtool{vtkUnstructuredToFoam}!ユーティリティ}%
\index{ユーティリティ!vtkUnstructuredToFoam@\OFtool{vtkUnstructuredToFoam}}%
 \OFtool{vtkUnstructuredToFoam} & VTK/ParaViewにより作られた
 アスキーの .vtk(旧形式)ファイルを変換する \\
\index{writeMeshObj@\OFtool{writeMeshObj}!ユーティリティ}%
\index{ユーティリティ!writeMeshObj@\OFtool{writeMeshObj}}%
 \OFtool{writeMeshObj} & メッシュのデバッグのため:
 たとえばjavaviewで見れる,三つの別々のOBJファイルとしてメッシュを書く \\
 \\
 \multicolumn{2}{l}{メッシュの操作} \\
 \hline
 \tblstrut
\index{attachMesh@\OFtool{attachMesh}!ユーティリティ}%
\index{ユーティリティ!attachMesh@\OFtool{attachMesh}}%
 \OFtool{attachMesh} & 指定されたメッシュ修正ユーティリティによって
 トポロジ的に独立したメッシュを付加する \\
\index{autoPatch@\OFtool{autoPatch}!ユーティリティ}%
\index{ユーティリティ!autoPatch@\OFtool{autoPatch}}%
 \OFtool{autoPatch} & ユーザが指定した角度に基づいて外部面をパッチに分割する \\
\index{checkMesh@\OFtool{checkMesh}!ユーティリティ}%
\index{ユーティリティ!checkMesh@\OFtool{checkMesh}}%
 \OFtool{checkMesh} & メッシュの妥当性をチェックする \\
\index{createBaffles@\OFtool{createBaffles}!ユーティリティ}%
\index{ユーティリティ!createBaffles@\OFtool{createBaffles}}%
 \OFtool{createBaffles} &
 内部面を境界面にする.
 \OFtool{mergeOrSplitBaffles}と異なり,点の複製はしない. \\
\index{createPatch@\OFtool{createPatch}!ユーティリティ}%
\index{ユーティリティ!createPatch@\OFtool{createPatch}}%
 \OFtool{createPatch} & 選択した境界面の外部にパッチを作成する.
 面は既存のパッチかfaceSetから選択する \\
\index{deformedGeom@\OFtool{deformedGeom}!ユーティリティ}%
\index{ユーティリティ!deformedGeom@\OFtool{deformedGeom}}%
 \OFtool{deformedGeom} & \OFtool{polyMesh}を変位場\OFkeyword{U}と
 引数として与えられた尺度因子により変形させる \\
\index{flattenMesh@\OFtool{flattenMesh}!ユーティリティ}%
\index{ユーティリティ!flattenMesh@\OFtool{flattenMesh}}%
 \OFtool{flattenMesh} & 2次元デカルトメッシュの前後の面を平らにする \\
\index{insideCells@\OFtool{insideCells}!ユーティリティ}%
\index{ユーティリティ!insideCells@\OFtool{insideCells}}%
 \OFtool{insideCells} & 面の内側に中心があるセルを抽出する.
 面は閉じていて単一である必要がある \\
\index{mergeMeshes@\OFtool{mergeMeshes}!ユーティリティ}%
\index{ユーティリティ!mergeMeshes@\OFtool{mergeMeshes}}%
 \OFtool{mergeMeshes} & 二つのメッシュを合体させる \\
\index{mergeOrSplitBaffles@\OFtool{mergeOrSplitBaffles}!ユーティリティ}%
\index{ユーティリティ!mergeOrSplitBaffles@\OFtool{mergeOrSplitBaffles}}%
 \OFtool{mergeOrSplitBaffles} &
 同じ点を共有する面(バッフル)を探索し,
 それらの面をマージ,もしくは点を複製する. \\
\index{mirrorMesh@\OFtool{mirrorMesh}!ユーティリティ}%
\index{ユーティリティ!mirrorMesh@\OFtool{mirrorMesh}}%
 \OFtool{mirrorMesh} & 与えられた面に対してメッシュの鏡映をつくる. \\
\index{moveDynamicMesh@\OFtool{moveDynamicMesh}!ユーティリティ}%
\index{ユーティリティ!moveDynamicMesh@\OFtool{moveDynamicMesh}}%
 \OFtool{moveDynamicMesh} & メッシュの移動とトポロジ変化のユーティリティ \\
\index{moveEngineMesh@\OFtool{moveEngineMesh}!ユーティリティ}%
\index{ユーティリティ!moveEngineMesh@\OFtool{moveEngineMesh}}%
 \OFtool{moveEngineMesh} & エンジン解析のためにメッシュを動かすソルバ \\
\index{moveMesh@\OFtool{moveMesh}!ユーティリティ}%
\index{ユーティリティ!moveMesh@\OFtool{moveMesh}}%
 \OFtool{moveMesh} & メッシュを動かすソルバ \\
\index{objToVTK@\OFtool{objToVTK}!ユーティリティ}%
\index{ユーティリティ!objToVTK@\OFtool{objToVTK}}%
 \OFtool{objToVTK} & obj線(面ではない)のファイルを読み込み,vtkに変換する \\
\index{orientFaceZone@\OFtool{orientFaceZone}!ユーティリティ}%
\index{ユーティリティ!orientFaceZone@\OFtool{orientFaceZone}}%
 \OFtool{orientFaceZone} & \OFkeyword{faceZone}の方向を修正する. \\
\index{polyDualMesh@\OFtool{polyDualMesh}!ユーティリティ}%
\index{ユーティリティ!polyDualMesh@\OFtool{polyDualMesh}}%
 \OFtool{polyDualMesh} &
 polyMeshの双対を求め,特徴線やパッチの辺を忠実に再現する. \\
\index{refineMesh@\OFtool{refineMesh}!ユーティリティ}%
\index{ユーティリティ!refineMesh@\OFtool{refineMesh}}%
 \OFtool{refineMesh} & 複数の方向にセルを細分化する. \\
\index{rernumberMesh@\OFtool{renumberMesh}!ユーティリティ}%
\index{ユーティリティ!rernumberMesh@\OFtool{renumberMesh}}%
 \OFtool{renumberMesh} & 行列の帯幅を狭くするためにセルの順番を付け直す.
 全ての時刻ディレクトリから全ての計算領域を読み込み,順番を付け直す \\
\index{rotateMesh@\OFtool{rotateMesh}!ユーティリティ}%
\index{ユーティリティ!rotateMesh@\OFtool{rotateMesh}}%
 \OFtool{rotateMesh} & メッシュおよび場を方向$\bm{n}_{1}$から方向$\bm{n}_{2}$へと回転させる \\
\index{setSet@\OFtool{setSet}!ユーティリティ}%
\index{ユーティリティ!setSet@\OFtool{setSet}}%
 \OFtool{setSet} & セル・面・点のセットやゾーンをインタラクティブに操作する \\
\index{setsToZones@\OFtool{setsToZones}!ユーティリティ}%
\index{ユーティリティ!setsToZones@\OFtool{setsToZones}}%
 \OFtool{setsToZones} & メッシュに
 \OFkeyword{pointZones}/\OFkeyword{faceZones}/\OFkeyword{cellZones}を,
 同様に名づけられた\OFkeyword{pointSets}/\OFkeyword{faceSets}/\OFkeyword{cellSets}から追加する \\
\index{singleCellMesh@\OFtool{singleCellMesh}!ユーティリティ}%
\index{ユーティリティ!singleCellMesh@\OFtool{singleCellMesh}}%
 \OFtool{singleCellMesh} &
 全てのフィールドを読み込み,内側の面が全て取り除かれたメッシュ (singleCellFvMesh)
 に変換して\OFregion{singleMesh}領域に書き出す. \\
\index{splitMesh@\OFtool{splitMesh}!ユーティリティ}%
\index{ユーティリティ!splitMesh@\OFtool{splitMesh}}%
 \OFtool{splitMesh} & 内部の面の外面を作ることでメッシュを分割する.\OFtool{attachDetach}を用いる \\
\index{splitMeshRegions@\OFtool{splitMeshRegions}!ユーティリティ}%
\index{ユーティリティ!splitMeshRegions@\OFtool{splitMeshRegions}}%
 \OFtool{splitMeshRegions} & メッシュを複数の領域に分割する \\
\index{stitchMesh@\OFtool{stitchMesh}!ユーティリティ}%
\index{ユーティリティ!stitchMesh@\OFtool{stitchMesh}}%
 \OFtool{stitchMesh} & メッシュを縫う \\
\index{subsetMesh@\OFtool{subsetMesh}!ユーティリティ}%
\index{ユーティリティ!subsetMesh@\OFtool{subsetMesh}}%
 \OFtool{subsetMesh} & \OFtool{cellSet}に基づいたメッシュ領域を選択する \\
\index{topoSet@\OFtool{topoSet}!ユーティリティ}%
\index{ユーティリティ!topoSet@\OFtool{topoSet}}%
 \OFtool{topoSet} & ディクショナリによって
 \OFkeyword{faceSets}/\OFkeyword{cellSets}/\OFkeyword{pointSets}を操作する. \\
\index{transformPoints@\OFtool{transformPoints}!ユーティリティ}%
\index{ユーティリティ!transformPoints@\OFtool{transformPoints}}%
 \OFtool{transformPoints} & 平行移動,回転,拡大・縮小のオプションにしたがって,
 \OFpath{polyMesh}ディレクトリのメッシュの点を変形させる \\
\index{zipUpMesh@\OFtool{zipUpMesh}!ユーティリティ}%
\index{ユーティリティ!zipUpMesh@\OFtool{zipUpMesh}}%
 \OFtool{zipUpMesh} & 有効な形をもった全ての多面体のセルが閉じていることを確実にするために,
 ぶら下がった頂点をもつメッシュを読み込み,セルを閉じる \\
 \\
 \multicolumn{2}{l}{その他のメッシュ・ツール} \\
 \hline
 \tblstrut
\index{autoRefineMesh@\OFtool{autoRefineMesh}!ユーティリティ}%
\index{ユーティリティ!autoRefineMesh@\OFtool{autoRefineMesh}}%
 \OFtool{autoRefineMesh} & 境界面付近のセルを細分化するユーティリティ \\
\index{collapseEdges@\OFtool{collapseEdges}!ユーティリティ}%
\index{ユーティリティ!collapseEdges@\OFtool{collapseEdges}}%
 \OFtool{collapseEdges} & 短い辺をつぶし,また複数の辺を結合して一つの線分にする \\
\index{combinePatchFaces@\OFtool{combinePatchFaces}!ユーティリティ}%
\index{ユーティリティ!combinePatchFaces@\OFtool{combinePatchFaces}}%
 \OFtool{combinePatchFaces} & 同じセル内でパッチの重複した面をチェックし結合する.
 重複した面は,例えば細分化された隣接セルが削除されたり,
 同じセルに属する4面が取り残された結果として現れる. \\
\index{modifyMesh@\OFtool{modifyMesh}!ユーティリティ}%
\index{ユーティリティ!modifyMesh@\OFtool{modifyMesh}}%
 \OFtool{modifyMesh} & メッシュ要素を操作する \\
\index{PDRMesh@\OFtool{PDRMesh}!ユーティリティ}%
\index{ユーティリティ!PDRMesh@\OFtool{PDRMesh}}%
 \OFtool{PDRMesh} & PDRタイプのシミュレーションのための
 メッシュおよび場の調整ユーティリティ \\
\index{refineHexMesh@\OFtool{refineHexMesh}!ユーティリティ}%
\index{ユーティリティ!refineHexMesh@\OFtool{refineHexMesh}}%
 \OFtool{refineHexMesh} & セルを$2 \times 2 \times 2$に分割して
 六面体メッシュを細分化する \\
\index{refinementLevel@\OFtool{refinementLevel}!ユーティリティ}%
\index{ユーティリティ!refinementLevel@\OFtool{refinementLevel}}%
 \OFtool{refinementLevel} &
 細分化されたデカルト・メッシュの細分化レベルを判別する.
 スナップの前に実行すること \\
\index{refineWallLayer@\OFtool{refineWallLayer}!ユーティリティ}%
\index{ユーティリティ!refineWallLayer@\OFtool{refineWallLayer}}%
 \OFtool{refineWallLayer} & パッチに隣接するセルを細分化するユーティリティ \\
\index{removeFaces@\OFtool{removeFaces}!ユーティリティ}%
\index{ユーティリティ!removeFaces@\OFtool{removeFaces}}%
 \OFtool{removeFaces} & 面を削除し両隣のセルを結合するユーティリティ \\
\index{selectCells@\OFtool{selectCells}!ユーティリティ}%
\index{ユーティリティ!selectCells@\OFtool{selectCells}}%
 \OFtool{selectCells} & 面との関連でセルを選択する \\
\index{splitCells@\OFtool{splitCells}!ユーティリティ}%
\index{ユーティリティ!splitCells@\OFtool{splitCells}}%
 \OFtool{splitCells} & 平面でセルを分割するユーティリティ \\
 \\
 \multicolumn{2}{l}{画像の後処理} \\
 \hline
 \tblstrut
\index{ensightFoamReader@\OFtool{ensightFoamReader}!ユーティリティ}%
\index{ユーティリティ!ensightFoamReader@\OFtool{ensightFoamReader}}%
 \OFtool{ensightFoamReader} &
 変換せずにOpenFOAMのデータを直接読むための
 EnSightのライブラリ・モジュール \\
 \\
 \multicolumn{2}{l}{データ変換の後処理} \\
 \hline
 \tblstrut
\index{foamDataToFluent@\OFtool{foamDataToFluent}!ユーティリティ}%
\index{ユーティリティ!foamDataToFluent@\OFtool{foamDataToFluent}}%
 \OFtool{foamDataToFluent} & OpenFOAMデータをFluent形式へ変換する \\
\index{foamToEnsight@\OFtool{foamToEnsight}!ユーティリティ}%
\index{ユーティリティ!foamToEnsight@\OFtool{foamToEnsight}}%
 \OFtool{foamToEnsight} & OpenFOAMデータをEnSight形式へ変換する \\
\index{foamToEnsightParts@\OFtool{foamToEnsightParts}!ユーティリティ}%
\index{ユーティリティ!foamToEnsightParts@\OFtool{foamToEnsightParts}}%
 \OFtool{foamToEnsightParts} &
 OpenFOAMデータをEnSight形式へ変換する.
 それぞれのセル・ゾーンとパッチに対してEnsightパーツが作られる \\
\index{foamToGMV@\OFtool{foamToGMV}!ユーティリティ}%
\index{ユーティリティ!foamToGMV@\OFtool{foamToGMV}}%
 \OFtool{foamToGMV} & OpenFOAMの出力をGMVで読めるファイルに変換する. \\
\index{foamToTecplot360@\OFtool{foamToTecplot360}!ユーティリティ}%
\index{ユーティリティ!foamToTecplot360@\OFtool{foamToTecplot360}}%
 \OFtool{foamToTecplot360} & Tecplotバイナリファイル形式のライタ. \\
\index{foamToTetDualMesh@\OFtool{foamToTetDualMesh}!ユーティリティ}%
\index{ユーティリティ!foamToTetDualMesh@\OFtool{foamToTetDualMesh}}%
 \OFtool{foamToTetDualMesh} & polyMeshをtetDualMeshに変換する \\
\index{foamToVTK@\OFtool{foamToVTK}!ユーティリティ}%
\index{ユーティリティ!foamToVTK@\OFtool{foamToVTK}}%
 \OFtool{foamToVTK} & レガシーなVTKファイル形式のライタ. \\
\index{smapToFoam@\OFtool{smapToFoam}!ユーティリティ}%
\index{ユーティリティ!smapToFoam@\OFtool{smapToFoam}}%
 \OFtool{smapToFoam} & STAR-CD SMAPデータファイルを
 OpenFOAMの計算領域の形式に変換する \\
 \\
 \multicolumn{2}{l}{速度場の後処理} \\
 \hline
 \tblstrut
\index{Co@\OFtool{Co}!ユーティリティ}%
\index{ユーティリティ!Co@\OFtool{Co}}%
 \OFtool{Co} & \OFkeyword{phi}場からCourant数$\nCo$を計算し,
 \OFclass{volScalarField}として書き出す \\
\index{enstrophy@\OFtool{enstrophy}!ユーティリティ}%
\index{ユーティリティ!enstrophy@\OFtool{enstrophy}}%
 \OFtool{enstrophy} & 速度場\OFkeyword{U}のエンストロフィを計算し,書き出す \\
\index{flowType@\OFtool{flowType}!ユーティリティ}%
\index{ユーティリティ!flowType@\OFtool{flowType}}%
 \OFtool{flowType} & 速度場\OFkeyword{U}のflowTypeを計算し,書き出す \\
\index{Lambda2@\OFtool{Lambda2}!ユーティリティ}%
\index{ユーティリティ!Lambda2@\OFtool{Lambda2}}%
 \OFtool{Lambda2} & 速度勾配テンソルの対称,
 非対称部分の正方形の合計のうち2番目に大きな固有値を計算し,書き出す \\
\index{Mach@\OFtool{Mach}!ユーティリティ}%
\index{ユーティリティ!Mach@\OFtool{Mach}}%
 \OFtool{Mach} & 各時刻の速度場\OFkeyword{U}から局所Mach数を計算し,オプション指定により書き出す \\
\index{Pe@\OFtool{Pe}!ユーティリティ}%
\index{ユーティリティ!Pe@\OFtool{Pe}}%
 \OFtool{Pe} & 流束\OFkeyword{phi}からペクレ数$\nPe$を計算し,
 その最大値,\OFclass{surfaceScalarField} \OFkeyword{Pef},および\OFclass{volScalarField} \OFkeyword{Pe}を書き出す \\
\index{Q@\OFtool{Q}!ユーティリティ}%
\index{ユーティリティ!Q@\OFtool{Q}}%
 \OFtool{Q} & 速度勾配テンソルの第2不変量を計算し,書き出す \\
\index{streamFunction@\OFtool{streamFunction}!ユーティリティ}%
\index{ユーティリティ!streamFunction@\OFtool{streamFunction}}%
 \OFtool{streamFunction} & 各時刻の速度場\OFkeyword{U}の流れ機能を計算し,書き出す \\
\index{uprime@\OFtool{uprime}!ユーティリティ}%
\index{ユーティリティ!uprime@\OFtool{uprime}}%
 \OFtool{uprime} & \OFkeyword{uprime} ($\sqrt{2k/3}$) のスカラ場を計算し,書き出す \\
\index{vorticity@\OFtool{vorticity}!ユーティリティ}%
\index{ユーティリティ!vorticity@\OFtool{vorticity}}%
 \OFtool{vorticity} & 速度場\OFkeyword{U}の渦度を計算し,書き出す \\
 \\
 \multicolumn{2}{l}{応力場の後処理} \\
 \hline
 \tblstrut
\index{stressComponents@\OFtool{stressComponents}!ユーティリティ}%
\index{ユーティリティ!stressComponents@\OFtool{stressComponents}}%
 \OFtool{stressComponents} &
 各時刻の応力テンソル\OFkeyword{sigma}の六つの要素の
 スカラ場を計算し,書き出す \\
 \\
 \multicolumn{2}{l}{スカラ場の後処理} \\
 \hline
 \tblstrut
\index{pPrime2@\OFtool{pPrime2}!ユーティリティ}%
\index{ユーティリティ!pPrime2@\OFtool{pPrime2}}%
 \OFtool{pPrime2} &
 各時刻の\OFkeyword{pPrime2} ($[p - \bar{p}]^{2}$) の
 スカラ場を計算し,書き出す \\
 \\
 \multicolumn{2}{l}{壁の後処理} \\
 \hline
 \tblstrut
\index{wallGradU@\OFtool{wallGradU}!ユーティリティ}%
\index{ユーティリティ!wallGradU@\OFtool{wallGradU}}%
 \OFtool{wallGradU} & 壁における\OFkeyword{U}の勾配を計算し,書き出す \\
\index{wallHeatFlux@\OFtool{wallHeatFlux}!ユーティリティ}%
\index{ユーティリティ!wallHeatFlux@\OFtool{wallHeatFlux}}%
 \OFtool{wallHeatFlux} &
 \OFkeyword{volScalarField}の境界面として
 全てのパッチに対する熱流束を計算し,書き出す.
 そして全ての壁について積分した熱流量も書き出す \\
\index{wallShearStress@\OFtool{wallShearStress}!ユーティリティ}%
\index{ユーティリティ!wallShearStress@\OFtool{wallShearStress}}%
 \OFtool{wallShearStress} &
 指定した時刻の全てのパッチに対し,乱流壁面せん断応力を計算して書き出す \\
\index{yPlus@\OFtool{yPlus}!ユーティリティ}%
\index{ユーティリティ!yPlus@\OFtool{yPlus}}%
 \OFtool{yPlus} &
 指定した時刻の全てのパッチについて,壁近傍セルの\OFkeyword{yPlus}を計算して書き出す.
 層流,LES,RAS,いずれにも使用可 \\
 \\
 \multicolumn{2}{l}{乱流の後処理} \\
 \hline
 \tblstrut
\index{createTurbulenceFields@\OFtool{createTurbulenceFields}!ユーティリティ}%
\index{ユーティリティ!createTurbulenceFields@\OFtool{createTurbulenceFields}}%
 \OFtool{createTurbulenceFields} & 乱流場を表すすべての変数を生成する \\
\index{R@\OFtool{R}!ユーティリティ}%
\index{ユーティリティ!R@\OFtool{R}}%
 \OFtool{R} & 現在の時間ステップについて,
 レイノルズ応力\OFkeyword{R}を計算して書き出す \\
 \\
 \multicolumn{2}{l}{パッチの後処理} \\
 \hline
 \tblstrut
\index{patchAverage@\OFtool{patchAverage}!ユーティリティ}%
\index{ユーティリティ!patchAverage@\OFtool{patchAverage}}%
 \OFtool{patchAverage} & 指定したフィールドの指定したパッチにわたる平均を計算する \\
\index{patchIntegrate@\OFtool{patchIntegrate}!ユーティリティ}%
\index{ユーティリティ!patchIntegrate@\OFtool{patchIntegrate}}%
 \OFtool{patchIntegrate} & 指定したフィールドの指定したパッチにわたる積分を計算する \\
 \\
 \multicolumn{2}{l}{ラグランジアン・シミュレーションの後処理} \\
 \hline
 \tblstrut
\index{particleTracks@\OFtool{particleTracks}!ユーティリティ}%
\index{ユーティリティ!particleTracks@\OFtool{particleTracks}}%
 \OFtool{particleTracks} &
 \OFrevision*{用語不明:tracked-parcel-type cloud}%
 パーセル追跡タイプの雲を使って計算されたケースの
 粒子の飛跡をVTKファイルに書き出す. \\
\index{steadyParticleTracks@\OFtool{steadyParticleTracks}!ユーティリティ}%
\index{ユーティリティ!steadyParticleTracks@\OFtool{steadyParticleTracks}}%
 \OFtool{steadyParticleTracks} &
 \OFrevision*{用語不明:steady-state cloud}%
 定常状態の雲を使って計算されたケースの
 粒子の飛跡をVTKファイルに書き出す.
 注意:使う前に(並列で計算しているなら)ケースを再構築しておく必要がある. \\
 \\
 \multicolumn{2}{l}{サンプリングの後処理} \\
 \hline
 \tblstrut
\index{probeLocations@\OFtool{probeLocations}!ユーティリティ}%
\index{ユーティリティ!probeLocations@\OFtool{probeLocations}}%
 \OFtool{probeLocations} & 位置での値を調べる \\
\index{sample@\OFtool{sample}!ユーティリティ}%
\index{ユーティリティ!sample@\OFtool{sample}}%
 \OFtool{sample} & 選択した補間スキーム,サンプリング・オプション,
 書き出しフォーマットに従って,フィールドのデータをサンプリングする. \\
 \\
 \multicolumn{2}{l}{様々な後処理} \\
 \hline
 \tblstrut
\index{dsmcFieldsCalc@\OFtool{dsmcFieldsCalc}!ユーティリティ}%
\index{ユーティリティ!dsmcFieldsCalc@\OFtool{dsmcFieldsCalc}}%
 \OFtool{dsmcFieldsCalc} & DSMC計算による広域的に平均化された場から,
 \OFkeyword{U}や\OFkeyword{T}といった集約的な場を計算する \\
\index{engineCompRatio@\OFtool{engineCompRatio}!ユーティリティ}%
\index{ユーティリティ!engineCompRatio@\OFtool{engineCompRatio}}%
 \OFtool{engineCompRatio} & 幾何的な圧縮比を計算する.
 BDCとTCDで体積を計算するので,
 バルブと非有効体積があるかどうか注意すること \\
\index{execFlowFunctionObjects@\OFtool{execFlowFunctionObjects}!ユーティリティ}%
\index{ユーティリティ!execFlowFunctionObjects@\OFtool{execFlowFunctionObjects}}%
 \OFtool{execFlowFunctionObjects} & 選択された時間セットに対して,
 選択されたディクショナリ (デフォルトでは\OFdictionary{system/controlDict}) で
 指定された関数オブジェクトのセットを実行する.
 代わりのディクショナリは\OFdictionary{system/}ディレクトリ内に置く. \\
% \index{foamCalc@\OFtool{foamCalc}!ユーティリティ}%
% \index{ユーティリティ!foamCalc@\OFtool{foamCalc}}%
%  \OFtool{foamCalc} & 指定された時刻におけるフィールド計算の汎用ユーティリティ. \\
\index{foamListTimes@\OFtool{foamListTimes}!ユーティリティ}%
\index{ユーティリティ!foamListTimes@\OFtool{foamListTimes}}%
 \OFtool{foamListTimes} & \OFclass{timeSelector}を使って時刻をリスト化する. \\
\index{pdfPlot@\OFtool{pdfPlot}!ユーティリティ}%
\index{ユーティリティ!pdfPlot@\OFtool{pdfPlot}}%
 \OFtool{pdfPlot} & 確率密度関数のグラフを生成する. \\
\index{postChannel@\OFtool{postChannel}!ユーティリティ}%
\index{ユーティリティ!postChannel@\OFtool{postChannel}}%
 \OFtool{postChannel} & チャンネル流計算のポストプロセスデータ \\
\index{ptot@\OFtool{ptot}!ユーティリティ}%
\index{ユーティリティ!ptot@\OFtool{ptot}}%
 \OFtool{ptot} &  時刻ごとに全圧を計算する \\
\index{wdot@\OFtool{wdot}!ユーティリティ}%
\index{ユーティリティ!wdot@\OFtool{wdot}}%
 \OFtool{wdot} &  時刻ごとにwdotを計算し,書き出す \\
\index{writeCellCentres@\OFtool{writeCellCentres}!ユーティリティ}%
\index{ユーティリティ!writeCellCentres@\OFtool{writeCellCentres}}%
 \OFtool{writeCellCentres} &  閾値制限してポストプロセスで使えるよう,
 格子中心の三つの座標値を\OFclass{volScalarField}として書き出す \\
 \\
 \multicolumn{2}{l}{面メッシュ (例えばSTL) ツール} \\
 \hline
 \tblstrut
\index{surfaceAdd@\OFtool{surfaceAdd}!ユーティリティ}%
\index{ユーティリティ!surfaceAdd@\OFtool{surfaceAdd}}%
 \OFtool{surfaceAdd} & 二つの面を加える.点を幾何学的に同化させる.
 三角形の重複や交差のチェックは行わない. \\
\index{surfaceAutoPatch@\OFtool{surfaceAutoPatch}!ユーティリティ}%
\index{ユーティリティ!surfaceAutoPatch@\OFtool{surfaceAutoPatch}}%
 \OFtool{surfaceAutoPatch} & 特性角度によって面をパッチにする.
 \OFtool{autoPatch}と同様. \\
\index{surfaceBooleanFeatures@\OFtool{surfaceBooleanFeatures}!ユーティリティ}%
\index{ユーティリティ!surfaceBooleanFeatures@\OFtool{surfaceBooleanFeatures}}%
 \OFtool{surfaceBooleanFeatures} & 二つの面についてのブーリアン演算のインタフェースのための
 \OFkeyword{extendedFeatureEdgeMesh}を生成する. \\
\index{surfaceCheck@\OFtool{surfaceCheck}!ユーティリティ}%
\index{ユーティリティ!surfaceCheck@\OFtool{surfaceCheck}}%
 \OFtool{surfaceCheck} & 幾何的・トポロジ的な面の品質をチェックする. \\
\index{surfaceClean@\OFtool{surfaceClean}!ユーティリティ}%
\index{ユーティリティ!surfaceClean@\OFtool{surfaceClean}}%
 \OFtool{surfaceClean} & - バッフルを除去,- 小さなエッジをつぶして三角形を除去,
 - 細長い三角形の頂点を底辺に射影してエッジに分解する. \\
\index{surfaceCoarsen@\OFtool{surfaceCoarsen}!ユーティリティ}%
\index{ユーティリティ!surfaceCoarsen@\OFtool{surfaceCoarsen}}%
 \OFtool{surfaceCoarsen} & `bunnylod' を使って面を粗くする. \\
\index{surfaceConvert@\OFtool{surfaceConvert}!ユーティリティ}%
\index{ユーティリティ!surfaceConvert@\OFtool{surfaceConvert}}%
 \OFtool{surfaceConvert} & ある面メッシュの書式を他のものに変換する. \\
\index{surfaceFeatureConvert@\OFtool{surfaceFeatureConvert}!ユーティリティ}%
\index{ユーティリティ!surfaceFeatureConvert@\OFtool{surfaceFeatureConvert}}%
 \OFtool{surfaceFeatureConvert} & \OFclass{edgeMesh}の書式との変換を行う. \\
\index{surfaceFeatureExtract@\OFtool{surfaceFeatureExtract}!ユーティリティ}%
\index{ユーティリティ!surfaceFeatureExtract@\OFtool{surfaceFeatureExtract}}%
 \OFtool{surfaceFeatureExtract} & 面要素を取り出し,ファイルに書き込む. \\
\index{surfaceFind@\OFtool{surfaceFind}!ユーティリティ}%
\index{ユーティリティ!surfaceFind@\OFtool{surfaceFind}}%
 \OFtool{surfaceFind} & 近くの面と頂点を見つける. \\
\index{surfaceHookUp@\OFtool{surfaceHookUp}!ユーティリティ}%
\index{ユーティリティ!surfaceHookUp@\OFtool{surfaceHookUp}}%
 \OFtool{surfaceHookUp} & 近接した開いたエッジを見つけて,
 それらに沿った面を縫い合わせる. \\
\index{surfaceInertia@\OFtool{surfaceInertia}!ユーティリティ}%
\index{ユーティリティ!surfaceInertia@\OFtool{surfaceInertia}}%
 \OFtool{surfaceInertia} & コマンドラインで指定された\OFclass{triSurface}の
 慣性テンソル・慣性主軸・慣性モーメントを計算する.
 ソリッドもしくは薄い殻の慣性が算出可能.\\
\index{surfaceLambdaMuSmooth@\OFtool{surfaceLambdaMuSmooth}!ユーティリティ}%
\index{ユーティリティ!surfaceLambdaMuSmooth@\OFtool{surfaceLambdaMuSmooth}}%
 \OFtool{surfaceLambdaMuSmooth} & lambda/muスムージングを用いて面を滑らかにする.
 ラプラシアンスムージングを取得 (以前の\OFtool{surfaceSmooth}の挙動) する際に,
 lambdaを緩和係数に,muをゼロにセットする. \\
\index{surfaceMeshConvert@\OFtool{surfaceMeshConvert}!ユーティリティ}%
\index{ユーティリティ!surfaceMeshConvert@\OFtool{surfaceMeshConvert}}%
 \OFtool{surfaceMeshConvert} & オプションで\OFclass{coordinateSystem}上での
 スケーリングや変形(回転・移動)を伴って,面のフォーマットを変換する. \\
\index{surfaceMeshConvertTesting@\OFtool{surfaceMeshConvertTesting}!ユーティリティ}%
\index{ユーティリティ!surfaceMeshConvertTesting@\OFtool{surfaceMeshConvertTesting}}%
 \OFtool{surfaceMeshConvertTesting} & ある面メッシュのフォーマットを他のものに変換する.
 ただし現時点では試験的な機能. \\
\index{surfaceMeshExport@\OFtool{surfaceMeshExport}!ユーティリティ}%
\index{ユーティリティ!surfaceMeshExport@\OFtool{surfaceMeshExport}}%
 \OFtool{surfaceMeshExport} & オプションで\OFclass{coordinateSystem}上での
 スケーリングや変形(回転・移動)を伴って,
 \OFclass{surfMesh}をさまざまなサードパーティの面フォーマットにエクスポートする. \\
\index{surfaceMeshImport@\OFtool{surfaceMeshImport}!ユーティリティ}%
\index{ユーティリティ!surfaceMeshImport@\OFtool{surfaceMeshImport}}%
 \OFtool{surfaceMeshImport} & オプションで\OFclass{coordinateSystem}上での
 スケーリングや変形(回転・移動)を伴って,
 さまざまなサードパーティの面フォーマットから\OFclass{surfMesh}にインポートする. \\
\index{surfaceMeshInfo@\OFtool{surfaceMeshInfo}!ユーティリティ}%
\index{ユーティリティ!surfaceMeshInfo@\OFtool{surfaceMeshInfo}}%
 \OFtool{surfaceMeshInfo} & 面メッシュに関するさまざまな情報 \\
\index{surfaceMeshTriangulate@\OFtool{surfaceMeshTriangulate}!ユーティリティ}%
\index{ユーティリティ!surfaceMeshTriangulate@\OFtool{surfaceMeshTriangulate}}%
 \OFtool{surfaceMeshTriangulate} & polyMeshからtriSurfaceを取り出す.
 アウトプットの面フォーマットに依存して,面を三角形にする.
 三角形の領域番号は\OFclass{polyMesh}のパッチ番号になる.
 オプションで名前のついたパッチのみを三角形にする. \\
\index{surfaceOrient@\OFtool{surfaceOrient}!ユーティリティ}%
\index{ユーティリティ!surfaceOrient@\OFtool{surfaceOrient}}%
 \OFtool{surfaceOrient} & ユーザが与えた「外側の」点に従って,法線を設定する.
 \texttt{-inside}を使うと,与えた点は内側とみなされる. \\
\index{surfacePointMerge@\OFtool{surfacePointMerge}!ユーティリティ}%
\index{ユーティリティ!surfacePointMerge@\OFtool{surfacePointMerge}}%
 \OFtool{surfacePointMerge} & 面上で,絶対距離以内にある点をマージする.
 絶対距離であることに注意. \\
\index{surfaceRedistributePar@\OFtool{surfaceRedistributePar}!ユーティリティ}%
\index{ユーティリティ!surfaceRedistributePar@\OFtool{surfaceRedistributePar}}%
 \OFtool{surfaceRedistributePar} & \OFclass{triSurface}を(再)配置する.
 分解されていない面またはすでに分解された面を,
 それぞれのプロセッサがそのメッシュにオーバラップする三角形全てをもつように再配置する. \\
\index{surfaceRefineRedGreen@\OFtool{surfaceRefineRedGreen}!ユーティリティ}%
\index{ユーティリティ!surfaceRefineRedGreen@\OFtool{surfaceRefineRedGreen}}%
 \OFtool{surfaceRefineRedGreen} &
 三角形の三辺全てを分割して精密化する (`red' 精密化).
 (精密化するようマークされていない)隣り合う三角形は半分にする (`green' 精密化).
 (R. Verfuerth, ``A review of a posteriori error estimation and
 adaptive mesh refinement techniques'', Wiley--Teubner, 1996) \\
\index{surfaceSplitByPatch@\OFtool{surfaceSplitByPatch}!ユーティリティ}%
\index{ユーティリティ!surfaceSplitByPatch@\OFtool{surfaceSplitByPatch}}%
 \OFtool{surfaceSplitByPatch} & triSurfaceの領域を個別のファイルに書きだす. \\
\index{surfaceSplitByTopology@\OFtool{surfaceSplitByTopology}!ユーティリティ}%
\index{ユーティリティ!surfaceSplitByTopology@\OFtool{surfaceSplitByTopology}}%
 \OFtool{surfaceSplitByTopology} & 面の邪魔板部分を全てはぎ取る. \\
\index{surfaceSplitNonManifolds@\OFtool{surfaceSplitNonManifolds}!ユーティリティ}%
\index{ユーティリティ!surfaceSplitNonManifolds@\OFtool{surfaceSplitNonManifolds}}%
 \OFtool{surfaceSplitNonManifolds} & 複雑に接続された面を,
 点を複製することで複雑に接続されたエッジにおいて分割しようとする.
 コンセプトを紹介すると,
\OFrevision*{?}%
 - borderEdgeは四つの面が接続しているエッジ,
 - borderPointeはちょうど二つのborderEdgeが接続している点,
 - borderLineはborderEdgeの接続リスト. \\
\index{surfaceSubset@\OFtool{surfaceSubset}!ユーティリティ}%
\index{ユーティリティ!surfaceSubset@\OFtool{surfaceSubset}}%
 \OFtool{surfaceSubset} & triSurfaceの必要な部分だけを選択する面の解析ツール.
 \OFtool{subsetMesh}に基づいている. \\
\index{surfaceToPatch@\OFtool{surfaceToPatch}!ユーティリティ}%
\index{ユーティリティ!surfaceToPatch@\OFtool{surfaceToPatch}}%
 \OFtool{surfaceToPatch} & 面を読み込み,メッシュに面の領域を適用する.
 難しい作業には\OFclass{boundaryMesh}を使う. \\
\index{surfaceTransformPoints@\OFtool{surfaceTransformPoints}!ユーティリティ}%
\index{ユーティリティ!surfaceTransformPoints@\OFtool{surfaceTransformPoints}}%
 \OFtool{surfaceTransformPoints} & 面を変形(拡大縮小・回転)する.
 \OFtool{transformPoints}と同様であるが対象は面. \\
 \\
 \multicolumn{2}{l}{並行処理} \\
 \hline
 \tblstrut
\index{decomposePar@\OFtool{decomposePar}!ユーティリティ}%
\index{ユーティリティ!decomposePar@\OFtool{decomposePar}}%
 \OFtool{decomposePar} & OpenFOAMの並列計算用に
 ケースのメッシュと計算領域を自動的に分割する \\
\index{reconstructPar@\OFtool{reconstructPar}!ユーティリティ}%
\index{ユーティリティ!reconstructPar@\OFtool{reconstructPar}}%
 \OFtool{reconstructPar} & OpenFOAMの並列計算用に
 分割されたケースのフィールドを再構築する \\
\index{reconstructParMesh@\OFtool{reconstructParMesh}!ユーティリティ}%
\index{ユーティリティ!reconstructParMesh@\OFtool{reconstructParMesh}}%
 \OFtool{reconstructParMesh} & 幾何情報のみを使ってメッシュを再構築する \\
\index{redistributePar@\OFtool{redistributePar}!ユーティリティ}%
\index{ユーティリティ!redistributePar@\OFtool{redistributePar}}%
 \OFtool{redistributePar} & \OFpath{decomposeParDict}ファイルの設定に従って
分割されたメッシュとフィールドを再分配する. \\
 \\
 \multicolumn{2}{l}{熱物理に関連したユーティリティ} \\
 \hline
 \tblstrut
\index{adiabaticFlameT@\OFtool{adiabaticFlameT}!ユーティリティ}%
\index{ユーティリティ!adiabaticFlameT@\OFtool{adiabaticFlameT}}%
 \OFtool{adiabaticFlameT} & 与えられた燃料の種類・燃焼していない気体の
 温度と平衡定数に対して断熱状態の炎の温度を計算する \\
\index{chemkinToFoam@\OFtool{chemkinToFoam}!ユーティリティ}%
\index{ユーティリティ!chemkinToFoam@\OFtool{chemkinToFoam}}%
 \OFtool{chemkinToFoam} & CHEMKIN 3の熱運動と反応のデータファイルを
 OpenFOAMのフォーマットに変換する \\
\index{equilibriumCO@\OFtool{equilibriumCO}!ユーティリティ}%
\index{ユーティリティ!equilibriumCO@\OFtool{equilibriumCO}}%
 \OFtool{equilibriumCO} & 一酸化炭素の平衡状態を計算する \\
\index{equilibriumFlameT@\OFtool{equilibriumFlameT}!ユーティリティ}%
\index{ユーティリティ!equilibriumFlameT@\OFtool{equilibriumFlameT}}%
 \OFtool{equilibriumFlameT} & 与えられた燃料の種類・燃焼していない気体の
 温度と平衡定数に対して酸素,水,二酸化炭素の分離の影響を考慮して
 平衡状態の炎の温度を計算する \\
\index{mixtureAdiabaticFlameT@\OFtool{mixtureAdiabaticFlameT}!ユーティリティ}%
\index{ユーティリティ!mixtureAdiabaticFlameT@\OFtool{mixtureAdiabaticFlameT}}%
 \OFtool{mixtureAdiabaticFlameT} & 与えられた混合・温度に対して
 断熱状態の炎の温度を計算する \\
 \\
 \multicolumn{2}{l}{様々なユーティリティ} \\
 \hline
 \tblstrut
\index{expandDictionary@\OFtool{expandDictionary}!ユーティリティ}%
\index{ユーティリティ!expandDictionary@\OFtool{expandDictionary}}%
 \OFtool{expandDictionary} & 引数として与えられたディクショナリを読み込み,
 マクロなどを展開した結果を標準出力に書き出す\\
\index{foamDebugSwitches@\OFtool{foamDebugSwitches}!ユーティリティ}%
\index{ユーティリティ!foamDebugSwitches@\OFtool{foamDebugSwitches}}%
 \OFtool{foamDebugSwitches} & すべてのライブラリのデバッグスイッチを書き出す \\
\index{foamFormatConvert@\OFtool{foamFormatConvert}!ユーティリティ}%
\index{ユーティリティ!foamFormatConvert@\OFtool{foamFormatConvert}}%
 \OFtool{foamFormatConvert} &
 \OFdictionary{controlDict}に指定された書式に従って,
 ケースに関わる\OFkeyword{IOobject}をすべて変換する \\
\index{foamHelp@\OFtool{foamHelp}!ユーティリティ}%
\index{ユーティリティ!foamHelp@\OFtool{foamHelp}}%
 \OFtool{foamHelp} & OpenFOAMのヘルプユーティリティまわりの
 トップレベルのラッパユーティリティ \\
\index{foamInfoExec@\OFtool{foamInfoExec}!ユーティリティ}%
\index{ユーティリティ!foamInfoExec@\OFtool{foamInfoExec}}%
 \OFtool{foamInfoExec} & ケースを調べ,標準出力に情報を書きだす. \\
\index{patchSummary@\OFtool{patchSummary}!ユーティリティ}%
\index{ユーティリティ!patchSummary@\OFtool{patchSummary}}%
 \OFtool{patchSummary} & 指定された時刻について,
 各パッチに対するフィールドと境界条件を書き出す
\end{longtable}
