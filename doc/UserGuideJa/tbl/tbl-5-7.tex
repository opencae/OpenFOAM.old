%#! platex UserGuideJa
\begin{tabularx}{\textwidth}{lXl}
 キーワード & 意味 & 例 \\
 \hline
\index{castellatedMesh@\string\OFkeyword{castellatedMesh}!キーワード}%
\index{キーワード!castellatedMesh@\string\OFkeyword{castellatedMesh}}%
 \OFkeyword{castellatedMesh} & ギザギザのメッシュを作成するかどうか & \OFkeyword{true} \\
\index{snap@\string\OFkeyword{snap}!キーワード}%
\index{キーワード!snap@\string\OFkeyword{snap}}%
 \OFkeyword{snap} & 表面のスナップの有無 & \OFkeyword{true} \\
\index{doLayers@\string\OFkeyword{doLayers}!キーワード}%
\index{キーワード!doLayers@\string\OFkeyword{doLayers}}%
 \OFkeyword{doLayers} & レイヤの追加の有無 & \OFkeyword{true} \\
\index{mergeTolerance@\string\OFkeyword{mergeTolerance}!キーワード}%
\index{キーワード!mergeTolerance@\string\OFkeyword{mergeTolerance}}%
 \OFkeyword{mergeTolerance} & 初期メッシュの有界ボックスの比として許容値をまとめる & \OFkeyword{1e-06} \\
\index{debug@\string\OFkeyword{debug}!キーワード}%
\index{キーワード!debug@\string\OFkeyword{debug}}%
 \OFkeyword{debug} & 中間メッシュと画面プリントの記述の制御 \\
 & 最終メッシュのみ記述 & \OFkeyword{0} \\
 & 中間メッシュの記述 & \OFkeyword{1} \\
 & 後処理のため\OFkeyword{cellLevel}を付けた\OFkeyword{volScalarField}を記述 & \OFkeyword{2} \\
 & \OFpath{.obj}ファイルとして現在の交点を記述 & \OFkeyword{4} \\
\index{geometry@\string\OFkeyword{geometry}!キーワード}%
\index{キーワード!geometry@\string\OFkeyword{geometry}}%
 \OFkeyword{geometry} & 表面に使用した全てのジオメトリのサブディクショナリ \\
\index{castellatedMeshControls@\string\OFkeyword{castellatedMeshControls}!キーワード}%
\index{キーワード!castellatedMeshControls@\string\OFkeyword{castellatedMeshControls}}%
 \OFkeyword{castellatedMeshControls} & 城壁メッシュ制御のサブディクショナリ \\
\index{snapControls@\string\OFkeyword{snapControls}!キーワード}%
\index{キーワード!snapControls@\string\OFkeyword{snapControls}}%
 \OFkeyword{snapControls} & 表面スナップ制御のサブディクショナリ \\
\index{addLayersControls@\string\OFkeyword{addLayersControls}!キーワード}%
\index{キーワード!addLayersControls@\string\OFkeyword{addLayersControls}}%
 \OFkeyword{addLayersControls} & レイヤ追加制御のサブディクショナリ \\
\index{meshQualityControls@\string\OFkeyword{meshQualityControls}!キーワード}%
\index{キーワード!meshQualityControls@\string\OFkeyword{meshQualityControls}}%
 \OFkeyword{meshQualityControls} & メッシュ特性制御のサブディクショナリ \\
 \hline
\end{tabularx}
