%#! uplatex UserGuideJa
\begin{tabularx}{\textwidth}{lXp{10zw}}
 \multicolumn{3}{l}{必須入力} \\
 \hline
 \tblstrut
\index{numberOfSubdomains@\string\OFkeyword{numberOfSubdomains}!キーワード}%
\index{キーワード!numberOfSubdomains@\string\OFkeyword{numberOfSubdomains}}%
 \OFkeyword{numberOfSubdomains} & サブドメインの総数 & $N$ \\
\index{method@\string\OFkeyword{method}!キーワード}%
\index{キーワード!method@\string\OFkeyword{method}}%
 \OFkeyword{method} & 分割方法 &
\index{simple@\string\OFkeyword{simple}!キーワードエントリ}%
\index{キーワードエントリ!simple@\string\OFkeyword{simple}}%
         \OFkeyword{simple}/\hfil\break
\index{hierarchical@\string\OFkeyword{hierarchical}!キーワードエントリ}%
\index{キーワードエントリ!hierarchical@\string\OFkeyword{hierarchical}}%
         \OFkeyword{hierarchical}/\hfil\break
\index{scotch@\string\OFkeyword{scotch}!キーワードエントリ}%
\index{キーワードエントリ!scotch@\string\OFkeyword{scotch}}%
         \OFkeyword{scotch}/
\index{metis@\string\OFkeyword{metis}!キーワードエントリ}%
\index{キーワードエントリ!metis@\string\OFkeyword{metis}}%
         \OFkeyword{metis}/
\index{manual@\string\OFkeyword{manual}!キーワードエントリ}%
\index{キーワードエントリ!manual@\string\OFkeyword{manual}}%
         \OFkeyword{manual}/ \\
 \\
 \multicolumn{3}{l}{\OFkeyword{simpleCoeffs}エントリ} \\
 \hline
 \tblstrut
\index{n@\string\OFkeyword{n}!キーワード}%
\index{キーワード!n@\string\OFkeyword{n}}%
 \OFkeyword{n} & $x$,$y$,$z$のサブドメイン数 & $(n_{x}, n_{y}, n_{z})$ \\
\index{delta@\string\OFkeyword{delta}!キーワード}%
\index{キーワード!delta@\string\OFkeyword{delta}}%
 \OFkeyword{delta} & セルのスキュー因数 & 一般的には,$10^{-3}$ \\
 \\
 \multicolumn{3}{l}{\OFkeyword{hierarchicalCoeffs}エントリ} \\
 \hline
 \tblstrut
 \OFkeyword{n} & $x$,$y$,$z$のサブドメイン数 & $(n_{x}, n_{y}, n_{z})$ \\
 \OFkeyword{delta} & セルのスキュー因数 & 一般的には,$10^{-3}$ \\
\index{order@\string\OFkeyword{order}!キーワード}%
\index{キーワード!order@\string\OFkeyword{order}}%
 \OFkeyword{order} & 分割の順序 & \OFkeyword{xyz}/\OFkeyword{xzy}/\OFkeyword{yzx}... \\
 \\
 \multicolumn{3}{l}{%
\index{scotchCoeffs@\string\OFkeyword{scotchCoeffs}!キーワード}%
\index{キーワード!scotchCoeffs@\string\OFkeyword{scotchCoeffs}}%
 \OFkeyword{scotchCoeffs}エントリ} \\
 \hline
 \tblstrut
\index{processorWeights@\string\OFkeyword{processorWeights}!キーワード}%
\index{キーワード!processorWeights@\string\OFkeyword{processorWeights}}%
 \parbox[t]{10em}{\rule{0pt}{9pt}
                  \OFkeyword{processorWeights}\par
                  (省略可)}
 & プロセッサへのセルの割当の重み係数の一覧.
     例:\verb|<wt1>|はプロセッサ1の重み係数.
     重みは規格化され,
     どんな範囲の値も取ることが可能.
     & (\verb|<wt1>|...\verb|<wtN>|) \\
\index{strategy@\string\OFkeyword{strategy}!キーワード}%
\index{キーワード!strategy@\string\OFkeyword{strategy}}%
 \OFkeyword{strategy}
 & 分割の戦略.省略可能であり複雑
     &  \\
 \\
 \multicolumn{3}{l}{%
\index{manualCoeffs@\string\OFkeyword{manualCoeffs}!キーワード}%
\index{キーワード!manualCoeffs@\string\OFkeyword{manualCoeffs}}%
 \OFkeyword{manualCoeffs}エントリ} \\
 \hline
 \tblstrut
 \OFkeyword{dataFile} & プロセッサへのセルの割当のデータを含むファイル名 & \texttt{"<fileName>"} \\
 \\
 \multicolumn{3}{l}{分散型データの入力(省略可)---\autoref{ssec:3.4.3}参照} \\
 \hline
 \tblstrut
\index{distributed@\string\OFkeyword{distributed}!キーワード}%
\index{キーワード!distributed@\string\OFkeyword{distributed}}%
 \OFkeyword{distributed} & データが複数のディスクに分散しているかどうか & \OFkeyword{yes}/\OFkeyword{no} \\
\index{roots@\string\OFkeyword{roots}!キーワード}%
\index{キーワード!roots@\string\OFkeyword{roots}}%
 \OFkeyword{roots} & ケースディレクトリへのルートパス.
     例:\verb|<rt1>|はノード1へのルートパス
     & (\verb|<rt1>|...\verb|<rtN>|) \\
\end{tabularx}
