%#! uplatex UserGuideJa
\begin{tabular}{ll}
 種類 & 意味 \\
 \hline
 \tblstrut
\index{patch@\OFkeyword{patch}!キーワードエントリ}%
\index{キーワードエントリ!patch@\OFkeyword{patch}}%
 \OFkeyword{patch} & 一般的なパッチ \\
\index{symmetryPlane@\OFkeyword{symmetryPlane}!キーワードエントリ}%
\index{キーワードエントリ!symmetryPlane@\OFkeyword{symmetryPlane}}%
 \OFkeyword{symmetryPlane} & 対称面 \\
\index{empty@\OFkeyword{empty}!キーワードエントリ}%
\index{キーワードエントリ!empty@\OFkeyword{empty}}%
 \OFkeyword{empty} & 2次元形状の前後の面 \\
\index{wedge@\OFkeyword{wedge}!キーワードエントリ}%
\index{キーワードエントリ!wedge@\OFkeyword{wedge}}%
 \OFkeyword{wedge} & 軸対称形状のための,くさび型の前後 \\
\index{cyclic@\OFkeyword{cyclic}!キーワードエントリ}%
\index{キーワードエントリ!cyclic@\OFkeyword{cyclic}}%
 \OFkeyword{cyclic} & 周期境界面 \\
\index{wall@\OFkeyword{wall}!キーワードエントリ}%
\index{キーワードエントリ!wall@\OFkeyword{wall}}%
 \OFkeyword{wall} & 壁面(乱流の壁関数に使用) \\
\index{processor@\OFkeyword{processor}!キーワードエントリ}%
\index{キーワードエントリ!processor@\OFkeyword{processor}}%
 \OFkeyword{processor} & 並列計算時のプロセッサ間の境界 \\
 \hline
\end{tabular}
