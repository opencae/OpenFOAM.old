%#! platex ProgrammersGuideJa
\chapter{OpenFOAMの使用例}
\label{chap:3}


\section{円柱まわりの流れ}
\label{sec:3.1}

\subsection{問題設定}
\label{ssec:3.1.1}

\subsection{\OFtool{potentialFoam}について}
\label{ssec:3.1.2}

\subsection{メッシュ生成}
\label{ssec:3.1.3}

\subsection{境界条件と初期条件}
\label{ssec:3.1.4}

\subsection{ケースの実行}
\label{ssec:3.1.5}

\subsection{解析解の導出}
\label{ssec:3.1.6}

\subsection{課題}
\label{ssec:3.1.7}


\section{後ろ向きステップの定常乱流}
\label{sec:3.2}

\subsection{問題設定}
\label{ssec:3.2.1}

\subsection{メッシュ生成}
\label{ssec:3.2.2}

\subsection{境界条件と初期条件}
\label{ssec:3.2.3}

\subsection{ケースの制御}
\label{ssec:3.2.4}

\subsection{ケースの実行とポスト処理}
\label{ssec:3.2.5}


\section{前向きステップの超音速流れ}
\label{sec:3.3}

\subsection{問題設定}
\label{ssec:3.3.1}

\subsection{メッシュ生成}
\label{ssec:3.3.2}

\subsection{ケースの実行}
\label{ssec:3.3.3}

\subsection{課題}
\label{ssec:3.3.4}


\section{加圧された水タンクの減圧}
\label{sec:3.4}

\subsection{問題設定}
\label{ssec:3.4.1}

\subsection{メッシュ生成}
\label{ssec:3.4.2}

\subsection{実行の準備}
\label{ssec:3.4.3}

\subsection{ケースの実行}
\label{ssec:3.4.4}

\subsection{メッシュの改良による解の改善}
\label{ssec:3.4.5}


\section{液体の磁性流体流れ}
\label{sec:3.5}

\subsection{問題設定}
\label{ssec:3.5.1}

\subsection{メッシュ生成}
\label{ssec:3.5.2}

\subsection{ケースの実行}
\label{ssec:3.5.3}
